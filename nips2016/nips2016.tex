\documentclass{article}

% if you need to pass options to natbib, use, e.g.:
% \PassOptionsToPackage{numbers, compress}{natbib}
% before loading nips_2016
%
% to avoid loading the natbib package, add option nonatbib:
% \usepackage[nonatbib]{nips_2016}

\usepackage{nips2016}
%\usepackage[final]{nips_2016}

\usepackage[utf8]{inputenc} % allow utf-8 input
\usepackage[T1]{fontenc}    % use 8-bit T1 fonts
\usepackage{hyperref}       % hyperlinks
\usepackage{url}            % simple URL typesetting
\usepackage{booktabs}       % professional-quality tables
\usepackage{amsfonts}       % blackboard math symbols
\usepackage{amssymb}
\usepackage{nicefrac}       % compact symbols for 1/2, etc.
\usepackage{microtype}      % microtypography

\title{Convolutional Neural Networks for Graphs}

% The \author macro works with any number of authors. There are two
% commands used to separate the names and addresses of multiple
% authors: \And and \AND.
%
% Using \And between authors leaves it to LaTeX to determine where to
% break the lines. Using \AND forces a line break at that point. So,
% if LaTeX puts 3 of 4 authors names on the first line, and the last
% on the second line, try using \AND instead of \And before the third
% author name.

\author{
  Michaël Defferrard \\ %\thanks{further info} \\
  %Department of Electrical Engineering \\
  %École Polytechnique Fédérale de Lausanne (EPFL) \\
  EPFL \\
  Lausanne, Switzerland \\
  \texttt{michael.defferrard@epfl.ch} \\
  \And % \AND
  Xavier Bresson \\
  EPFL \\
  Lausanne, Switzerland \\
  \texttt{xavier.bresson@epfl.ch} \\
  \And
  Pierre Vandergheynst \\
  EPFL \\
  Lausanne, Switzerland \\
  \texttt{pierre.vandergheynst@epfl.ch} \\
}

\usepackage[acronym]{glossaries}
\newacronym{SVD}{SVD}{Singular Value Decomposition}
\newacronym{SGD}{SGD}{Stochastic Gradient Descent}
\newacronym{PSD}{PSD}{positive semidefinite}

\DeclareMathOperator*{\diag}{diag}
\DeclareMathOperator*{\argmin}{arg\,min}
\DeclareMathOperator*{\spn}{span}
\renewcommand{\L}{\mathcal{L}}
\newcommand{\G}{\mathcal{G}}
\renewcommand{\H}{\mathcal{H}}
\newcommand{\K}{\mathcal{K}}
\newcommand{\V}{\mathcal{V}}
\newcommand{\E}{\mathcal{E}}
\newcommand{\W}{\mathcal{W}}
\newcommand{\R}{\mathbb{R}}
\newcommand{\N}{\mathbb{N}}
\newcommand{\Xh}{\hat{X}}
\newcommand{\Yh}{\hat{Y}}
\newcommand{\st}{\ \text{s.t.} \,}
\newcommand{\norm}[1]{\left\| #1 \right\|}

\begin{document}
% \nipsfinalcopy is no longer used

\maketitle

\begin{abstract}
This paper introduces Graph Neural Network (GNN), a novel framework to process
data on irregular, i.e.~non-Euclidean domains modelized as graphs. Even recent
developments in Deep Learning who use graph embeddings don't make use of the
graph structure per se.
\end{abstract}

\section{Introduction}\label{introduction}

Large networks allow to learn complex relationships. Although they
overfit easily if the training set is small because there is a lot of
parameters to learn.

Convolutional neural networks (CNN) offer a great reduction of
parameters on regular Euclidean domain. This work proposes to generalize
it to the discrete model of manifolds, graphs. Such an approach was
proposed by \citep{bruna_spectral_2013, henaff_deep_2015}, we will
refine it with proper graph signal processing tools.

This paper introduces a framework to use CNN with signals defined on
weighted graphs. Our work leverages spectral methods, which are much
faster than a spatial approach would be, in the same sense as doing
convolutions as multiplication in the Fourier domain (with the help of
the FFT) is much faster than multiplying patches on an image.

Another option to do clustering or regression on data graphs is to first
embed the graph and then use the embedding as additional features.

\section{Related work}\label{related-work}

Comparison with fully connected NN (Reuters used in
\citep{srivastava_dropout_2014})

\begin{itemize}
\item
  + less parameters via convolution
\item
  ? accuracy
\end{itemize}

Comparison with first generation graph CNN \citep{henaff_deep_2015}.

\begin{itemize}
\item
  + better accuracy
\item
  = same number of parameters
\item
  - graph construction vs supervised estimation
\item
  Theoretical:

  \begin{itemize}
  \item
    + proper spectral graph signal processing tools
  \item
    + kNN graph (sparse weight matrix) vs fully connected graph. SVD
    complexity for Fourier basis is O(n\^{}3), we avoid it.
  \item
    + METIS vs naive agglomeration
  \item
    + Chebyshev polynomials vs splines to approximate filters Filtering
    from \(O(n^2)\) to \(O(|E| K)\) where \(K\) is the polynomial order
    approximately \(O(n)\) if the weight matrix is sparse, i.e.~grows
    linearly with \(N\) This addresses the first limitation mentioned in
    \citep{henaff_deep_2015}.
  \end{itemize}
\end{itemize}

\section{Model}\label{model}

\subsection{Spectral Graph Theory}\label{spectral-graph-theory}

A graph \(\G = (\V, \E, \W)\) is defined by a set \(V\) of \(|V| = M\)
nodes and a set \(E\) of weighted edges. The connectivity of the graph
is captured by the adjacency matrix \(W \in \R^{M \times M}\) which
entry \(W_{i,j}\) denotes the weight of the edge \((v_i, v_j) \in E\)
which connects the vertex \(v_i \in V\) to \(v_j \in V\). It is set to
\(0\) if the vertices are not connected, i.e. \((v_i, v_j) \notin E\).
Assuming an undirected graph, \(W\) is a symmetric matrix. A graph
signal is any signal \(x \in \R^M\) defined on the vertices of \(G\).

The combinatorial graph Laplacian is defined as
\[ \L^c := D - W \in \R^{M \times M}, \] where \(D \in \R^{M \times M}\)
is the diagonal degree matrix defined as \(D_{i,i} := \sum_j W_{i,j}\).
Note that it is a difference operator such that
\[ (\L^c x)_i = \sum_j W_{i,j} (x_i - x_j). \] The normalized graph
Laplacian is defined as
\[ \L^n := I_M - D^{-1/2} W D^{-1/2} \in \R^{M \times M}, \] where
\(I_M \in \R^{M \times M}\) is the identity matrix. Finally, the
random-walk graph Laplacian is defined as
\[ \L^{rw} := I_M - D^{-1} W \in \R^{M \times M}. \] Note that this work
is independent of the chosen graph Laplacian \(\L\), which can be any of
\(\L^c\), \(\L^n\) or \(\L^{rw}\).

In analogy to the real line Fourier Transform, a Fourier basis is given
by the eigenvectors of the Laplacian
\[ \L u_i = \lambda_i u_i \st \norm{u_i}_2=1,
\ u_i \in \R^M, \ i = 0, \ldots, M-1, \] with their associated
eigenvalues \(\lambda_i\). Assuming the graph is connected, we may order
the vector \(\lambda := [\lambda_0, \ldots, \lambda_{M-1}]^T \in \R^M\)
of eigenvalues such that
\[ 0 = \lambda_{min} = \lambda_0 < \lambda_1 \leq \ldots \leq
\lambda_{M-1} = \lambda_{max}. \] As the Laplacian is a real symmetric
and \gls{PSD} matrix, the eigenvalues are real and positive, and the
eigenvectors are orthonormal. The Laplacian is indeed diagonalized by
the Fourier basis \(U := [u_0, \ldots, u_{M-1}] \in \R^{M \times M}\)
such that
\begin{equation} \L = U \Lambda U^T, \label{eq:lap_diag}\end{equation}
where \(\Lambda := \diag(\lambda) \in \R^{M \times M}\) denotes a
diagonal matrix of eigenvalues and \(U^T\) is the matrix transpose of
\(U\). See \citep{chung_spectral_1997} for details on spectral graph
theory.

The Graph Fourier Transform \(\hat{x} \in \R^M\) of any graph signal
\(x\) is given by \[ \hat{x} = U^T x =
[\langle u_0, x \rangle, \ldots, \langle u_{M-1}, x \rangle]^T, \] where
\(\langle \cdot , \cdot \rangle\) denotes the standard \(\ell_2\) inner
product \citep{shuman_emerging_2013}. The inverse transform is then
given by \[
x = U \hat{x}. \]

It follows that any signal \(\hat{x}\) can be filtered in the spectral
domain by \[ \hat{y} = g_\theta(\Lambda) \hat{x}, \] where the operator
\(g_\theta(\Lambda) \in \R^{M \times M}\), a matrix function, yields a
diagonal matrix of \(M\) Fourier coefficients parametrized by
\(\theta\). The filtering operation can equivalently take place in the
vertex domain as \[ y = U g_\theta(\Lambda) U^T x = g_\theta(\L) x \]
where the operator
\(g_\theta(\L) = g_\theta(U \Lambda U^T) = U g_\theta(\Lambda) U^T \in \R^{M \times M}\)
is akin to a convolution with the filter \(Ug_\theta(\Lambda)\) in the
vertex domain. Note that the filtering operation is defined as a
multiplication in the spectral domain because a convolution cannot be
defined in the vertex domain \citep{shuman_emerging_2013}.

\subsection{Graph Filter Learning}\label{graph-filter-learning}

Given a graph \(G\), a set
\(X = [x_0, \ldots, x_{N-1}] \in \R^{M \times N}\) of \(N\) source graph
signals of dimensionality \(M\) and their associated set
\(Y = [y_0, \ldots, y_{N-1}] \in \R^{M \times N}\) of target signals, we
want to learn the parameters \(\theta \in \R^M\) of a graph filter
\(g_\theta(\Lambda) := \diag(\theta)\) such as to minimize the convex
mean square error \begin{equation} L =
\frac{1}{N} \sum_{i=0}^{N-1} \norm{ g_\theta(\L) x_i - y_i }_2^2 =
\frac{1}{N} \norm{ U \diag(\theta) U^T X - Y }_F^2, \label{eq:loss}\end{equation}
where \(\norm{\cdot}_2^2\) denotes the squared \(\ell_2\) norm and
\(\norm{\cdot}_F^2\) the squared Frobenius norm.

Note that, while being the most flexible definition of
\(g_\theta(\Lambda)\), the independence of the Fourier coefficients
\(\theta\) from the eigenvalues \(\Lambda\) does omit all information
about frequencies. As we know that the lower frequencies are more
important for clustering {[}ref NCut{]}, we could have designed a
parametric filter with less parameters.

Rewriting eq.~\ref{eq:loss} in the spectral domain while decomposing it
w.r.t. the scalar coefficients \(\theta_i\) gives \[ L =
\frac{1}{N} \norm{ \diag(\theta) U^T X - U^T Y }_F^2 =
\frac{1}{N} \sum_{i=0}^{M-1} \norm{\theta_i\Xh_{i,\cdot}-\Yh_{i,\cdot}}_2^2, \]
where \(\Xh = U^TX \in \R^{M \times N}\) and
\(\Yh = U^TY \in \R^{M \times N}\) are the spectral representations of
the signals \(X\) and \(Y\). The gradient for each coefficient is then
given by \[ \nabla_{\theta_i} L =
\frac{2}{N} ( \theta_i \Xh_{i,\cdot} - \Yh_{i,\cdot} ) \Xh^T_{\cdot,i} \]
and can be rewritten in a vectorized form as
\begin{equation} \nabla_{\theta} L =
\frac{2}{N} \diag \left( (\diag(\theta) \Xh - \Yh) \Xh^T \right) =
\frac{2}{N} \left( ( \theta 1_N^T \odot \Xh - \Yh ) \odot \Xh \right) 1_N,
\label{eq:gradient}\end{equation} where \(1_N\) denotes a unit vector of
length \(N\) and \(\odot\) the element-wise Hadamard product. The second
form avoids the computation of the useless off-diagonal elements.

A closed-form solution is given by the optimality condition
\(\nabla_{\theta}L=0\) such that
\begin{equation} \theta^* = \argmin_\theta L =
(\Xh \odot \Yh) 1_N \oslash (\Xh \odot \Xh) 1_N, \label{eq:direct}\end{equation}
where \(\oslash\) denotes an element-wise division. Note that this
method is impractical for large \(M\) and \(N\) (sufficiently large for
\(X\) and \(Y\) to not fit in memory). A \gls{SGD} based on
eq.~\ref{eq:gradient} is then necessary.

There are however two major computational drawbacks to this optimization
process: 1) the Fourier transform \(\Xh=U^TX\) of a set of signals costs
\(O(M^2N)\) operations and 2) the eigenvalue decomposition
\(\L = U \Lambda U^T\) costs \(O(M^3)\). The total cost of filtering is
thus \(O(M^2 \max(M,N))\), a problem already stated in
\citep{henaff_deep_2015}. We propose to overcome these computational
issues using efficient numerical approximations such as the Chebyshev
polynomials or the Lanczos algorithm.

\subsubsection{Chebyshev
Parametrization}\label{chebyshev-parametrization}

The main idea, to avoid the Fourier basis, is twofold: 1) parametrize
the filter, in the spectral domain, as a truncated Chebyshev expansion
and 2) recursively compute the Chebyshev polynomials from the Laplacian.
This approximate method for spectral graph filtering was first proposed
in \citep{hammond_wavelets_2011} for a fast wavelet transform on graphs.

Recall that the Chebyshev polynomial \(T_k(x)\) of order \(k\) may be
generated by the stable recurrence relation
\(T_k(x) = 2x T_{k-1}(x) - T_{k-2}(x)\) with \(T_0 = 1\) and
\(T_1 = x\). These polynomials form an orthogonal basis for
\(L^2([-1,1], dy / \sqrt{1-y^2})\), the Hilbert space of square
integrable functions with respect to the measure \(dy/\sqrt{1-y^2}\).

The graph filter \(g_\theta(\Lambda)\) can thus be constructed from the
truncated expansion
\[ g_\theta(\Lambda) := \sum_{k=0}^{K-1} \theta_k T_k(\tilde{\Lambda}) \]
of polynomial order \(K-1\), where the parameter \(\theta \in \R^K\),
\(K \ll M\), is a vector of Chebyshev coefficients and
\(T_k(\tilde{\Lambda}) \in \R^{M \times M}\) is the Chebyshev polynomial
of order \(k\) evaluated at
\(\tilde{\Lambda} := 2 \Lambda / \lambda_{max} - I_M \in \R^{M \times M}\),
a diagonal matrix of scaled eigenvalues (so that they lie in
\([-1,1]\)). While reducing the number of parameters from \(M\) to
\(K\), this parametrization enforces smoothness in the spectral domain,
which translates to localization in the vertex domain. It can indeed be
shown that \((\L^k)_{i,j}=0\) if the shortest-path between vertices
\(v_i\) and \(v_j\) is longer than \(k\) edges
\citep{hammond_wavelets_2011}, which limits the influence of a
\(K^\text{th}\) order filter to \(K\) hopes. This is often a desired
property, e.g.~to learn local features for classification.

To avoid the Fourier basis \(U\), we express the polynomials as
functions of the Laplacian \(\L\) instead of its eigenvalues \(\Lambda\)
using eq.~\ref{eq:lap_diag}, such that the filtering operator in the
vertex domain is given by
\begin{equation} g_\theta(\L) = U g_\theta(\Lambda) U^T =
\sum_{k=0}^{K-1} U \theta_k T_k(\tilde{\Lambda}) U^T =
\sum_{k=0}^{K-1} \theta_k T_k(\tilde{\L}), \label{eq:chebyshev}\end{equation}
where \(T_k(\tilde{\L}) \in \R^{M \times M}\) is the Chebyshev
polynomial of order \(k\) evaluated at the scaled Laplacian
\(\tilde{\L} := 2 \L / \lambda_{max} - I_M\). Note that the spectrum of
the normalized Laplacian is bounded by \(2\)
\citep{chung_spectral_1997}, such that the scaling can simply be
\(\tilde{\L} = \L - I_M\), tolerating some imprecision in the
approximation due to the loss of a fraction \(2-\lambda_{max}\) of the
domain.

Inserting eq.~\ref{eq:chebyshev} into eq.~\ref{eq:loss} we obtain
\begin{equation} L =
\frac{1}{N} \norm{ \sum_{k=0}^{K-1} \theta_k T_k(\tilde{\L}) X - Y }_F^2 =
\frac{1}{N} \norm{ \bar{X} \theta - \bar{y} }_2^2, \label{eq:loss_c}\end{equation}
where \(\bar{y} \in \R^{MN}\) is the vectorized matrix \(Y\) and the
\(k^\text{th}\) column of \(\bar{X} \in \R^{MN \times K}\) is the
vectorized matrix
\(\tilde{X}_k := T_k(\tilde{\L}) X \in \R^{M \times N}\). The gradient
is then given by \begin{equation} \nabla_\theta L =
\frac{2}{N} \bar{X}^T (\bar{X} \theta - \bar{y}). \label{eq:gradient_c}\end{equation}

While the system \(\bar{X} \theta - \bar{y}\) is largely over-determined
as \(K \ll MN\), a closed-form solution of eq.~\ref{eq:loss_c} is given
by the optimality condition \(\nabla_\theta L = 0\) so that
\begin{equation} \theta^* = \argmin_\theta L = \bar{X}^+ \bar{y}, \label{eq:direct_c}\end{equation}
where \(\bar{X}^+ = (\bar{X}^T\bar{X})^{-1} \bar{X}^T\) is the
pseudo-inverse of \(\bar{X}\). The computation of the inverse is fast
and stable for small \(K\). Alternative approximate solution methods for
larger order \(K\) include the computation of the pseudo-inverse with
\gls{SVD}, the use of a least-square solver or a gradient descent scheme
based on eq.~\ref{eq:gradient_c}.

Using the recurrence
\[ \tilde{X}_k = 2\tilde{\L} \tilde{X}_{k-1} - \tilde{X}_{k-2} \] with
\(\tilde{X}_0 = X\) and \(\tilde{X}_1 = \tilde{\L} X\), the computation
of \(\bar{X}\) given \(X\) costs \(O(K|E|N) = O(KMN)\) operations if the
number of edges \(|E|\) is proportional to the number of nodes \(M\),
e.g.~for kNN graphs. As the cost of the product \(\bar{X}\theta\) is
similar, the entire filtering operation has a computational cost of
\(O(KMN)\) whereas the straightforward implementation using the Fourier
basis has a cost of \(O(M^2\max(M,N)\). When applying multiple filters
to the same set of signals, as is the case with \gls{SGD}, one may save
computations by storing and querying \(\bar{X}\) instead of \(X\), at
the expense of \(K\) times the memory.

\subsubsection{Lanczos Parametrization}\label{lanczos-parametrization}

Another applicable parametrization is based on the Lanczos algorithm, an
adaptation of the power iteration to find the largest or smallest
eigenvalues and corresponding eigenvectors of a linear system. It was
first introduced for fast graph filtering in
\citep{susnjara_accelerated_2015}.

The algorithm, described in
\citep{gallopoulos_efficient_1992, susnjara_accelerated_2015},
constructs an orthonormal basis
\(V = [v_0, \ldots, v_{K-1}] \in \R^{M \times K}\) of the Krylov
subspace \(\mathcal{K}_K(\L,x) = \spn\{ x, \L x, \ldots, \L^{K-1} x \}\)
and a tri-diagonal matrix \(H = V^T \L V \in \R^{K \times K}\) with a
computational cost of \(O(K |E|)\). Note that for large
\(K \gtrapprox 30\), the original iterative algorithm may loose the
basis orthogonality such that a necessary orthogonalization step will
increase the complexity \citep{susnjara_accelerated_2015}. Filtering the
signal \(x\) with \(g_\theta(\L)\) can then be approximated by an order
\(K-1\) polynomial as \begin{equation} y = g_\theta(\L) x \approx
V g_\theta(H) V^T x = V Q g_\theta(\Sigma) Q^T V^T x \label{eq:lanczos}\end{equation}
where \(Q \Sigma Q^T = H\) is the eigendecomposition of \(H\). There
exist fast methods for the eigendecomposition of symmetric tri-diagonal
matrices {[}{]}. As for the Chebyshev approximation, this construction
enforces smoothness in the spectral domain.

Inserting eq.~\ref{eq:lanczos} with a parametrized filter
\(g_\theta(\Sigma) := \diag(\theta)\), \(\theta \in \R^K\), \(K \ll M\),
into eq.~\ref{eq:loss} gives \[ L = \frac{1}{N} \sum_{n=0}^{N-1}
\norm{V_n Q_n \diag(\theta) Q_n^T V_n^T x_n - y_n}_2^2 =
\frac{1}{N} \norm{\diag(\theta) \hat{X} - \hat{Y}}_F^2 \] where
\(\hat{X} = [\hat{x}_0, \ldots, \hat{x}_{N-1}] \in \R^{K \times N}\),
\(\hat{Y} = [\hat{y}_0, \ldots, \hat{y}_{N-1}] \in \R^{K \times N}\) and
\(\hat{x}_n = Q_n^T V_n^T x_n \in \R^K\),
\(\hat{y}_n = Q_n^T V_n^T y_n \in \R^K\) are approximate representations
of the signals \(x_n\), \(y_n\) in the orthonormal basis
\(V_n Q_n \in \R^{M \times K}\) of \(\mathcal{K}_K(\L,x_n)\). Similarly
to eq.~\ref{eq:loss}, the gradient is given by eq.~\ref{eq:gradient} and
a closed-form solution by eq.~\ref{eq:direct}.

The expression eq.~\ref{eq:lanczos} can be simplified by setting
\(v_0 := x / \norm{x}_2\) (the first basis vector can be set to an
arbitrary unit length vector) such that \(V^T x = \norm{x}_2 e_1\) and
\[ V g_\theta(H) V^T x =
\norm{x}_2 V Q \diag(\theta) Q^T e_1 =
\norm{x}_2 V Q \diag(q) \theta \] where \(e_1 \in \R^K\) denotes the
first unit vector and \(q = Q^T e_1 \in \R^K\) is the first row of
\(Q\). Inserting into eq.~\ref{eq:loss} gives \[ L =
\frac{1}{N} \sum_{n=0}^{N-1} \norm{\tilde{X}_n \theta - y_n}_2^2 =
\frac{1}{N} \sum_{k=0}^{K-1} \norm{\bar{X}_{\cdot,k} \theta_k - \bar{y}}_2^2 =
\frac{1}{N} \norm{\bar{X} \theta - \bar{y} }_2^2 \] where
\(\bar{y} \in \R^{NM}\) is the vectorized matrix \(Y\) and
\(\bar{X} := [\tilde{X}_0, \ldots, \tilde{X}_{N-1}]^T \in \R^{NM \times K}\)
is a stack of \(N\) matrices
\(\tilde{X}_n := \norm{x_n}_2 V_n Q_n \diag(q_n) \in \R^{M \times K}\)
where \(V_n\), \(Q_n\) and \(q_n\) are derived from \(x_n\). The second
form (in terms of independent coefficients \(\theta_k\)) is valid
because \(\bar{X}\) is orthogonal, i.e \(\bar{X}^T \bar{X}\) is a
diagonal matrix, so that the solution eq.~\ref{eq:direct_c} can be
written as \begin{equation} \theta^* = \argmin_\theta L =
\underbrace{\sum_{n=0}^{N-1} \Big( \norm{x_n}_2 \diag(q_n) \Big)^{-2}}
_{(\bar{X}^T\bar{X})^{-1}} \bar{X}^T \bar{y} \label{eq:direct_l}\end{equation}
where
\(\bar{X}^T \bar{y} = [\langle \bar{X}_{\cdot,0}, \bar{y} \rangle, \ldots, \langle \bar{X}_{\cdot,K-1}, \bar{y} \rangle]^T \in \R^K\)
is the projection of \(\bar{y}\) onto the \(K\) basis vectors
\(\bar{X}_{\cdot,k} \in \R^{MN}\). While this closed-form evaluation is
fast and stable, a gradient descent scheme can be used with
eq.~\ref{eq:gradient_c}.

While the time and space complexities are similar, this method has two
advantages over the Chebyshev approximation: 1) it does not require the
normalization of the Laplacian spectrum (thus the estimation of
\(\lambda_{max}\)) and 2) the optimization is easier as the parameters
\(\theta_k\) are independent of each other thanks to the orthogonal
basis. By easier we mean that 1) the closed-form solution
eq.~\ref{eq:direct_l} does not involve the inversion of a large matrix
and 2) \gls{SGD} converges to a better solution with less iterations.
See \citep{susnjara_accelerated_2015} for a discussion of the
approximation quality and a comparison with the Chebyshev approximation.

\subsection{Graph coarsening}\label{graph-coarsening}

The graph partitioning problem has been widely studied. Applications
include VLSI design, load balancing for parallel computations, network
analysis, and optimal scheduling. The goal is to partition the vertices
of a graph into a certain number of disjoint sets of approximately the
same size so that a cut metric is minimized. This problem is NP-complete
even for several restricted classes of graphs, and there is no constant
factor approximation algorithm for general graphs {[}Bui and Jones
1992{]}. While notable developments in exact algorithms and heuristics
have been done, only the introduction of the general-purpose multilevel
methods during the 1990s has provided a breakthrough in efficiency and
quality {[}Safro 2009 2012{]}. Graph coarsening, where the problem
instance is gradually mapped to smaller ones to reduce the original
complexity, i.e., the graph underlying the problem is compressed, is the
first step of these multilevel methods.

Existing multilevel algorithms for combinatorial optimization problems
(such as k-partitioning, linear ordering, clustering, and segmentation)
can be divided into two classes: contraction-based schemes and algebraic
multigrid (AMG)-inspired schemes.

Alternative standard algorithms for graph coarsening include the Kron
reduction {[}{]}, Label Propagation {[}{]}, spectral clustering {[}{]}
or a sampling {[}Nath uncertainty{]}. A standard approach for graph
coarsening is the maximal match, as used by METIS for graph partitioning
{[}?{]}.

The assumption behind METIS is that a (near) optimal partitioning on a
coarser graph implies a good partitioning in the finer graph. In
general, it only holds true when the degree of nodes in the graph is
bounded by a constant {[}Karypis 95{]}. Some real-life graphs,
e.g.~social networks or WWW, are right-skewed, i.e.~there exists hub
vertices which have very large degrees. These graphs follow a power-law
degree distribution.

\section{Applications}\label{applications}

\subsection{Regression}\label{regression}

Solving a linear regression problem on graph signals is akin to minimize
eq.~\ref{eq:loss} where \(Y\) is the dependent variable and \(X\) the
independent observations.

\subsection{Representation learning}\label{representation-learning}

Representation learning is the problem of finding an appropriate (to
some criteria) representation \(y_i\) of a sample \(x_i\) defined as
\[ \min_{A,Y} \norm{ AX - Y }_F^2 + \tau P(Y) \] where
\(A \in \R^{M \times M}\) is some operator and \(P(Y)\) is some prior on
\(Y\). Take \(P(Z)=\norm{Z}_F^2\) and you get a Thikhonov prior. Take
\(P(Y) = \norm{Y}_1\) and you get sparse coding (with dictionary
learning).

\subsection{Classification}\label{classification}

There are dependent vectors
\(X = \{x_i\}_{i=0}^{N-1} \in \R^{M \times N}\) we want to predict from
observations \(Y = \{y_i\}_{i=0}^{N-1} \in \R^{M \times N}\).

In the classification case, \(y_i \in \N\) is a discrete number which
represents a class, e.g. \(y_i=-1\) indicates that sample \(x_i\)
belongs to class 1 and \(y_i=+1\) to class 2.

\subsubsection{Representation learning}\label{representation-learning-1}

Three ways to reduce the number of parameters:

\begin{enumerate}
\def\labelenumi{\arabic{enumi}.}
\item
  Graph coarsening
\item
  No weights: loss of spatiality (convolutional only)
\item
  Non-linear activation function
\end{enumerate}

But \(\theta\) and \(w\) are redundant, so
\[ L = \norm{\overline{\bar{X} \theta}^T 1_{FM} - y}_2^2 \]
\[ \bar{X} \in \R^{NM \times K} \] \[ \theta \in \R^{K \times F} \]

\[ \nabla_\theta L = \bar{X}^T (\bar{X} \theta - z) \]
\[ \nabla_\theta L = (\overline{\bar{X} \theta}^T 1_{FM} - y) \]

\[ \bar{X} 1_{FM} (\overline{\bar{X} \theta}^T 1_{FM} - y) \]
\[ (\bar{X} 1_{FM} \overline{\bar{X}}^T 1_{FM})^{-1} \bar{X} 1_{FM} y \]

\subsubsection{Linear classification}\label{linear-classification}

Empirical risk minimization (statistical learning).

The \(\ell_2\) regularized least-square model, also known as ridge
regression, is defined as \begin{equation} L =
\frac{1}{N} \sum_{i=0}^{N-1} |f_w(x_i) - y_i|^2 + \tau \norm{f}_\H^2,
\label{eq:rls}\end{equation} where \(f_w \in \H\) is some classification
function parametrized by \(w\) and \(\H\) is an Hilbert space.

Introducing the linear classifier \(f(x) = w^T x\) into eq.~\ref{eq:rls}
gives \[ L = \frac{1}{N} \norm{X^T w - y}_2^2  + \tau \norm{w}_2^2, \]
where \(w \in \R^M\) are the weights to learn. Note that there is no
bias because the data can be centered around zero (we know the lower and
upper bounds on luminosity values).

Gradient
\begin{equation} \nabla_w L = \frac{2}{N} X (X^T w - y) + 2 \tau w \label{eq:gradient_rls}\end{equation}

Optimality condition \(\nabla L = 0\)
\begin{equation} w^* = \argmin_w \nabla L =
\left(X X^T + \tau N I_M \right)^{-1} X y \label{eq:sol_rls}\end{equation}

\subsubsection{Feature learning}\label{feature-learning}

Introducing graph filter learning as a scheme for feature learning gives
the objective \[ L = \frac{1}{N} \sum_{i=0}^{N-1}
\Big| f_w \Big( A_\theta(\L) x_i \Big) - y_i \Big|^2 +
\tau_R \norm{f_w}_\H^2, \] where the filter bank
\(A_\theta(\L) = \bigcup_{i=0}^{F-1} g_{\theta_i}(\L) \in \R^{FM \times M}\)
is a block-diagonal matrix composed of \(F\) filters
\(g_{\theta_i}(\L)\) which extract different features.

Using a linear classifier and the Lanczos parametrization introduced in
\ref{lanczos-parametrization} gives \begin{equation} L =
\frac{1}{N} \norm{\overline{\bar{X} \theta}^T w - y}_2^2 +
\tau_R \norm{w}_2^2, \label{eq:rls_gfl}\end{equation} where
\(w \in \R^{FM}\) are the weights to learn,
\(\bar{X} \in \R^{NM \times K}\) is the Lanczos basis described in
\ref{lanczos-parametrization}, \(\theta \in \R^{K \times F}\) are the
\(K\) coefficients of the \(F\) filters and
\(\overline{\bar{X} \theta} \in \R^{FM \times N}\) is a rearrangement of
the matrix \(\bar{X} \theta \in \R^{NM \times F}\). Note that there is
now \(F\) times more weigts to learn, i.e. \(w \in \R^{FM}\).

\(\bar{X} \theta \in \R^{NM \times F}\)

\[ \frac{2}{N} \bar{X} w (\overline{\bar{X} \theta}^T w - y) \]

\[ (\bar{X} w \overline{\bar{X}}^T w)^{-1} \bar{X}wy \]

\subsubsection{Splitting scheme}\label{splitting-scheme}

The splitting scheme \[ L =
\frac{1}{N} \norm{Z^T w - y}_2^2 +
\frac{\tau_F}{N} \norm{\overline{\bar{X} \theta} - Z}_F^2 +
\tau_R \norm{w}_2^2 \] may be used to minimize the non-convex objective
eq.~\ref{eq:rls_gfl} for \(\theta\) and \(w\), where
\(Z = [z_0, \ldots, z_{N-1}] \in \R^{FM \times N}\) are the signals in
feature space. Solving the overall non-convex problem requires to
iteratively solve the following three convex sub-problems:

\begin{enumerate}
\def\labelenumi{\arabic{enumi}.}
\item
  The gradient and solution to \(\min_\theta L\) are given by
  eq.~\ref{eq:gradient_c} and eq.~\ref{eq:direct_l}, substituting
  \(\bar{y}\) by \(\bar{Z} \in \R^{NM \times F}\), a rearrangement of
  \(Z\).
\item
  The gradient of \(\min_Z L\) is \[ \nabla_Z L =
  \frac{2}{N} w (w^T Z - y^T) + \frac{2\tau_F}{N} (Z - \overline{\bar{X} \theta}) \]
  and the closed-form solution is \[ Z^* = \argmin_Z L =
  \left( \tau_F I_{FM} + w w^T \right)^{-1}
  \left( \tau_F \overline{\bar{X} \theta} + w y^T \right). \]
\item
  The gradient and solution to \(\min_w L\) are given by
  eq.~\ref{eq:gradient_rls} and eq.~\ref{eq:sol_rls}, substituting \(X\)
  by \(Z\).
\end{enumerate}

\subsubsection{Graph coarsening}\label{graph-coarsening-1}

To reduce again the number of parameters in \(w\), the graph should be
coarsened to trade feature resolution at the price of vertex resolution.

\subsubsection{Kernel methods}\label{kernel-methods}

The same can be done for kernel methods.

We want to learn a classification function
\(f(x) = \sum_i \K(x,x_i) \alpha_i \in \R\), where \(\K(\cdot,\cdot)\)
is some kernel function, which maps a sample \(x\) to a label \(y\). The
loss function to minimize over the training set is then given by
\[ L = \sum_i |f(x_i) - y_i|^2 + \tau \norm{f}_{H_\K}^2 =
\norm{ \K \alpha - y }_2^2 + \tau \alpha^T \K \alpha, \] where
\(\K_{i,j} = \K(x_i, x_j)\) is the kernel function evaluated at all
training samples and \(\alpha \in \R^N\) are the parameters.

\[ \nabla_\alpha L = \K (\K \alpha - y) + \tau \K \alpha \] \(\K\) is
symmetric.

\[ \alpha^* = \argmin_\alpha L = (\K + \tau I_N)^{-1} y \]

\subsubsection{Feature learning}\label{feature-learning-1}

A general formulation of the loss for a classification problem is given
by \[ \min_{\theta_A, \theta_h} \left\{ L = \frac{1}{N} \sum_{i=0}^{N-1}
\left| h_{\theta_h}(A_{\theta_A}(\L) x_i) - y_i \right|^2 +
\tau_A \Omega_A(A_{\theta_A}) + \tau_h \Omega_h(h_{\theta_h}) \right\} \]
where \(A\) is an operator which transforms the input signals into a
suitable, i.e.~learned, representation and \(h\) is a classification
function. \(\Omega_A\) and \(\Omega_h\) are regularizations, \(\tau_A\)
and \(\tau_h\) are hyper-parameters.

The SVM classification function \(h \in H_\K\) is defined as
\[ h_{\theta_h}(x) = \sum_{i=0}^{N-1} \alpha_i \K(x, x_i) + \beta,
\ \theta_h = (\alpha_i, \beta), \] where \(\K(\cdot,\cdot)\) is some
kernel function. It is regularized by
\[ \Omega_h = \norm{h}_{H_\K}^2. \]

In our graph filters learning setting, the mapping
\(A \in \R^{FM \times M}\) is given by the filter bank
\[ A_{\theta_A}(\L) = \bigcup_{f=0}^{F-1} g_{\theta_f}(\L),
\ \theta_A = (\theta_f), \] a block-diagonal matrix composed of \(F\)
filters, which extracts multiple features from the input signal.

While the optimization complexity would increase a lot if we were to
minimize for the model complexity \(K\), we can easily include a penalty
of the form
\[ \Omega_A = \sum_f \sum_i |\theta_f(\lambda_i) w(\lambda_i)|^2 \]
where \(w(\lambda_i) \sim 1 / \lambda_i\).

Learning complexity: there is \(FK\) parameters in \(\theta_A\) and
\(N+1\) parameters in \(\theta_h\).

\subsection{Going deep}\label{going-deep}

Define convolutional neural networks on irregular domains by learning
multiple layers of graph filters.

Image of the architecture.

\section{Experiments}\label{experiments}

\subsection{Filter learning}\label{filter-learning}

Example from the Juypiter notebook.

Ground truth filter. where \(c_\ell = g(\lambda_\ell)\) is the evaluated
filter.

(randomly generated) graph signals and a filter \(g(\lambda)\) for some
\(t\), we want to learn the filter coefficients \(C\), a diagonal
matrix, such that \[\min_C \sum_{i=1}^N \norm{ U^T C U x_i - y_i }_2^2\]
where \(U\) are the eigenvectors of the graph Laplacian \(L\) and
\(y_i = x_i * g + \varepsilon, \varepsilon \sim \mathcal{N}(0,\epsilon)\)
is the target signal (the filtered signal \(x_i\) with additive Gaussian
noise).

\subsubsection{Non-parametrized filter}\label{non-parametrized-filter}

\subsubsection{Spline Parametrization}\label{spline-parametrization}

The authors of \citep{bruna_spectral_2013, henaff_deep_2015} learned
graph filters parametrized by cubic splines.

\subsubsection{Chebyshev
Parametrization}\label{chebyshev-parametrization-1}

\subsubsection{Lanczos Parametrization}\label{lanczos-parametrization-1}

\subsection{Classification}\label{classification-1}

\subsubsection{Sampled MNIST}\label{sampled-mnist}

\subsubsection{Text documents}\label{text-documents}

\section{Discussion}\label{discussion}

\section{Conclusion and Outlook}\label{conclusion-and-outlook}

That was awesome.

\subsubsection*{Acknowledgments}

Thanks ...

\section*{References}


\end{document}
